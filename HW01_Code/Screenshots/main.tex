% AAAI 2024 LaTeX Template for CS 4820/5820 Homework 1
\documentclass[letterpaper]{article}
\usepackage{aaai24}
\usepackage{times}
\usepackage{helvet}
\usepackage{courier}
\usepackage{graphicx}
\usepackage{booktabs}
\usepackage{amsmath}
\usepackage{listings}
\usepackage{xcolor}
\usepackage{multirow}
\usepackage{array}

% Code listing style
\lstset{
    basicstyle=\ttfamily\small,
    breaklines=true,
    frame=single,
    language=Python,
    showstringspaces=false,
    numbers=left,
    numberstyle=\tiny\color{gray}
}

\title{CS 4820/5820 Homework 1: Problem Solving and Search}
\author{Josh Manchester\\
University of Colorado Colorado Springs\\
josh.manchester@uccs.edu}

\begin{document}
\maketitle

\section{Part A: Intelligent Agents (1.0 pts)}

\subsection{Question 1: PEAS Descriptions}

\subsubsection{(a) Autonomous Delivery Drone}

\textbf{Performance:} The drone should minimize energy use and costs while successfully delivering packages on time.

\textbf{Environment:} It operates outdoors where there are buildings, different weather conditions, and GPS signals. There are also no-fly zones and specific areas for takeoff and landing.

\textbf{Actuators:} Propellers and motors for flight control, control surfaces for steering, mechanisms to keep it upright, grippers to hold packages, lights, and a speaker for alerts.

\textbf{Sensors:} GPS to know where it is, an altimeter for altitude, cameras to detect obstacles, and laser sensors to measure distances.

\subsubsection{(b) Chess-Playing Agent}

\textbf{Performance:} Win rate, how fast it makes moves, and whether the moves are strategic or just random.

\textbf{Environment:} A standard 8×8 chess board where all pieces are visible, and the rules are deterministic (always the same outcome for the same moves).

\textbf{Actuators:} Either a robotic arm to move physical pieces or digital commands if it's software-based.

\textbf{Sensors:} A vision system with cameras if it's watching a physical board, or direct access to the board state in a digital game. It might also have access to a database of chess openings.

\subsubsection{(c) Self-Driving Taxi}

\textbf{Performance:} How efficiently it picks up and drops off passengers, fuel efficiency, travel time, passenger safety, and customer satisfaction.

\textbf{Environment:} Real roads with other cars, pedestrians, traffic lights, weather, and different road conditions.

\textbf{Actuators:} Wheels, steering wheel, gas pedal, brake, lights, and horn.

\textbf{Sensors:} Cameras, LIDAR, ultrasonic sensors, speedometer, GPS, accelerometer, engine sensors, and an interface for passengers to enter their destination.

\subsection{Question 2: Agent Type Classification}

\subsubsection{(a) Autonomous Delivery Drone: Model-Based Reflex Agent}

The drone keeps track of where it is (GPS coordinates, altitude, obstacles, weather) and uses this information to make decisions based on what it's currently sensing. Even though it has a goal (deliver the package), it mostly reacts to what's happening right now using its internal model instead of planning out a bunch of different possible actions.

\subsubsection{(b) Chess-Playing Agent: Goal-Based / Utility-Based Agent}

A chess AI has to think ahead about how to win the game by planning sequences of moves. It can't just react to the current board - it needs to think about future positions and figure out if they get it closer to checkmate. More advanced versions are utility-based because they use evaluation functions to score positions and try to maximize their chances of winning.

\subsubsection{(c) Self-Driving Taxi: Utility-Based Agent}

The taxi has to balance a lot of different things that sometimes compete with each other. It can't just focus on one goal - instead, it has to optimize a utility function that considers passenger safety (most important), travel time, fuel efficiency, comfort, following traffic laws, and finding the best route.

\section{Part B: n-Puzzle Problem (3.5 pts)}

\subsection{Problem Formulation}

\textbf{States:} All the possible ways you can arrange the tiles on the board. For the 8-puzzle (3×3), there are 9!/2 = 181,440 states you can actually reach. For the 15-puzzle (4×4), there are 16!/2 ≈ 10.4 trillion reachable states.

\textbf{Initial State:} Any valid starting configuration that's actually solvable. You can tell if it's solvable by checking the parity of inversions in the tile sequence.

\textbf{Actions:} Slide a tile that's next to the blank space into the blank. Or you can think of it as moving the blank up, down, left, or right (if it's not at an edge).

\textbf{Transition Model:} Given a state and an action, you get a new state by swapping the blank with the adjacent tile.

\textbf{Goal Test:} Check if the tiles are in order: 1, 2, 3, ..., n with the blank at the end.

\textbf{Path Cost:} Each move costs 1. The optimal solution is the one with the fewest moves.

\subsection{Implementation Details}

I implemented all the algorithms in Python from scratch without using any graph libraries. Here are the main design choices I made:

\begin{itemize}
    \item \textbf{State Representation:} I used tuples of integers (basically the board flattened into one line) because tuples can be hashed, which makes sets and dictionaries way more efficient.
    \item \textbf{Neighbor Generation:} I find where the blank is, then generate valid moves based on whether it's at a board edge or not.
    \item \textbf{Parent Tracking:} I used a dictionary to map each state back to its parent so I could reconstruct the solution path.
    \item \textbf{Explored Set:} A set of states I've already visited so I don't get stuck in loops (this is graph search instead of tree search).
\end{itemize}

\subsection{BFS Results}

Breadth-First Search looks at all the nodes at one level before moving to the next level, using a FIFO queue. This guarantees you'll find the shortest solution for problems where each action costs the same.

\begin{table}[h]
\centering
\tiny
\begin{tabular}{lrrrr}
\toprule
\textbf{Puzzle} & \textbf{Trial} & \textbf{Moves} & \textbf{Expanded} & \textbf{Time (s)} \\
\midrule
\multirow{3}{*}{8-puzzle} & 1 & 10 & 747 & 0.0015 \\
& 2 & 2 & 3 & 0.0000 \\
& 3 & 14 & 2,953 & 0.0046 \\
\cmidrule{2-5}
& \textbf{Avg} & \textbf{8.67} & \textbf{1,234} & \textbf{0.0020} \\
\midrule
\multirow{3}{*}{15-puzzle} & 1 & 4 & 55 & 0.0001 \\
& 2 & 6 & 189 & 0.0004 \\
& 3 & 6 & 114 & 0.0002 \\
\cmidrule{2-5}
& \textbf{Avg} & \textbf{5.33} & \textbf{119} & \textbf{0.0002} \\
\bottomrule
\end{tabular}
\caption{BFS Performance Results}
\end{table}

\textbf{Analysis:} BFS found optimal solutions for all the test cases. It guarantees the shortest path because it expands nodes level by level. The downside is that memory usage grows exponentially with depth, which makes it less practical for really scrambled 15-puzzles. 

\textbf{Note on 15-puzzle:} I had to use very light scrambling (only 12 random moves) for the 15-puzzle because BFS wouldn't finish in a reasonable time with deeper scrambles. So these results don't really compare well to the 8-puzzle results - they're just showing that BFS works on easier 15-puzzle instances.

\subsection{DFS Results}

Depth-First Search goes as deep as possible first using a LIFO stack. It's complete for finite graphs, but it doesn't guarantee optimal solutions.

\begin{table}[h]
\centering
\tiny
\begin{tabular}{lrrrr}
\toprule
\textbf{Puzzle} & \textbf{Trial} & \textbf{Moves} & \textbf{Expanded} & \textbf{Time (s)} \\
\midrule
\multirow{3}{*}{8-puzzle} & 1 & 56,106 & 65,837 & 0.1363 \\
& 2 & 64,326 & 84,888 & 0.1646 \\
& 3 & 16,928 & 17,467 & 0.0358 \\
\cmidrule{2-5}
& \textbf{Avg} & \textbf{45,787} & \textbf{56,064} & \textbf{0.1123} \\
\midrule
\multirow{4}{*}{15-puzzle} & 1 & --- & 10M & 33.87 \\
& 2 & 1 & 1 & 0.0000 \\
& 3 & 1 & 1 & 0.0000 \\
& 4 & 1 & 1 & 0.0000 \\
\cmidrule{2-5}
& \textbf{Avg} & \textbf{1.00} & \textbf{1.00} & \textbf{0.0000} \\
\bottomrule
\end{tabular}
\caption{DFS Performance Results}
\end{table}

\textbf{Analysis:} DFS found really bad solutions for the 8-puzzle (45,787 moves compared to the optimal 8.67). This shows the main problem with DFS - it explores depth-first without caring about optimality, so it often wanders down really deep paths that don't lead to a solution. For the 15-puzzle, the first attempt hit my 10M node limit, then I got lucky with some trivially easy puzzles.

\textbf{Note on 15-puzzle:} DFS is basically unusable for the 15-puzzle. Even with minimal scrambling (only 3 random moves), it would hit the 10M node cap or just run forever. I had to use extremely easy starting positions just to get any results at all. The data here doesn't really tell you much about DFS performance - it just shows that DFS is impractical for this problem.

\subsection{IDS Results}

Iterative Deepening Search combines DFS's low memory usage with BFS's optimality by running depth-limited DFS with increasing depth limits.

\begin{table}[h]
\centering
\tiny
\begin{tabular}{lrrrr}
\toprule
\textbf{Puzzle} & \textbf{Trial} & \textbf{Moves} & \textbf{Expanded} & \textbf{Time (s)} \\
\midrule
\multirow{3}{*}{8-puzzle} & 1 & 26 & 12,892 & 0.0276 \\
& 2 & 10 & 1,096 & 0.0035 \\
& 3 & 8 & 254 & 0.0005 \\
\cmidrule{2-5}
& \textbf{Avg} & \textbf{14.67} & \textbf{4,747} & \textbf{0.0106} \\
\midrule
\multirow{3}{*}{15-puzzle} & 1 & 4 & 38 & 0.0001 \\
& 2 & 2 & 5 & 0.0000 \\
& 3 & 4 & 19 & 0.0001 \\
\cmidrule{2-5}
& \textbf{Avg} & \textbf{3.33} & \textbf{20.67} & \textbf{0.0001} \\
\bottomrule
\end{tabular}
\caption{IDS Performance Results}
\end{table}

\textbf{Analysis:} IDS found optimal solutions while using less memory than BFS. It does re-expand nodes at shallower depths, but because of exponential growth, most of the work happens at the deepest level anyway. The overhead is only about $b/(b-1)$ where $b$ is the branching factor (roughly 1.33× for $b$=4).

\textbf{Note on 15-puzzle:} I had to use lighter scrambling (only 8 random moves) for the 15-puzzle because IDS wouldn't complete in reasonable time otherwise. The 15-puzzle results here are from much easier starting positions than the 8-puzzle, so they're not really comparable.

\subsection{BDS Results}

Bidirectional Search runs two BFS searches at the same time (one forward from the start, one backward from the goal) that meet in the middle. This gives you $O(b^{d/2})$ complexity instead of $O(b^d)$.

\begin{table}[h]
\centering
\tiny
\begin{tabular}{lrrrr}
\toprule
\textbf{Puzzle} & \textbf{Trial} & \textbf{Moves} & \textbf{Expanded} & \textbf{Time (s)} \\
\midrule
\multirow{3}{*}{8-puzzle} & 1 & 14 & 161 & 0.0004 \\
& 2 & 10 & 57 & 0.0002 \\
& 3 & 12 & 120 & 0.0003 \\
\cmidrule{2-5}
& \textbf{Avg} & \textbf{12.00} & \textbf{112.67} & \textbf{0.0003} \\
\midrule
\multirow{3}{*}{15-puzzle} & 1 & 4 & 7 & 0.0000 \\
& 2 & 4 & 7 & 0.0000 \\
& 3 & 6 & 19 & 0.0001 \\
\cmidrule{2-5}
& \textbf{Avg} & \textbf{4.67} & \textbf{11.00} & \textbf{0.0000} \\
\bottomrule
\end{tabular}
\caption{BDS Performance Results}
\end{table}

\textbf{Analysis:} BDS was the most efficient uninformed search algorithm by far, expanding way fewer nodes than BFS (113 vs 1,234 for 8-puzzle). The big improvement comes from searching from both ends: $2 \times b^{d/2}$ is much much smaller than $b^d$.

\textbf{Note on 15-puzzle:} Like the other algorithms, I had to use minimal scrambling (only 10 random moves) for the 15-puzzle to get any results. BDS handles it better than the other uninformed algorithms, but these results are still from much easier puzzles than the 8-puzzle tests.

\subsection{Algorithm Comparison}

\begin{table}[h]
\centering
\tiny
\begin{tabular}{lcccr}
\toprule
\textbf{Alg} & \textbf{Opt?} & \textbf{Comp?} & \textbf{Time} & \textbf{Space} \\
\midrule
BFS & Yes & Yes & $O(b^d)$ & $O(b^d)$ \\
DFS & No & Yes* & $O(b^m)$ & $O(bm)$ \\
IDS & Yes & Yes & $O(b^d)$ & $O(bd)$ \\
BDS & Yes & Yes & $O(b^{d/2})$ & $O(b^{d/2})$ \\
\bottomrule
\multicolumn{5}{l}{\tiny *Finite graphs with cycle detection}
\end{tabular}
\caption{Uninformed Search Comparison}
\end{table}

\section{Part C: n-Queens Problem (4.0 pts)}

\subsection{Problem Formulation}

\textbf{State Space:} Each state is a configuration of $n$ queens on an $n \times n$ board. I used a constrained encoding where each column has exactly one queen, represented as a list where \texttt{state[c] = row} means the queen in column $c$ is at row \texttt{row}. This reduces the state space from $n^{2n}$ to $n^n$, which is a lot better.

\textbf{Initial State:} Random placement of queens with one per column in a random row.

\textbf{Actions:} Move a queen within its column to a different row.

\textbf{Goal Test:} No queens attack each other (zero conflicts). Two queens attack if they're in the same row or diagonal: $|r_1 - r_2| = |c_1 - c_2|$.

\textbf{Evaluation Function:} Minimize the number of attacking pairs. For the genetic algorithm fitness, I used: fitness = $\frac{n(n-1)}{2}$ - conflicts.

\subsection{Simulated Annealing Implementation}

SA is a probabilistic search that accepts worse moves to escape local optima. The acceptance probability $P = e^{\Delta E/T}$ decreases as the temperature $T$ cools down.

\textbf{Key Parameters:}
\begin{itemize}
    \item Initial temperature $T_0$: Controls how much exploration happens at the start
    \item Cooling rate: Geometric schedule $T \leftarrow T \times \alpha$ where $0 < \alpha < 1$
    \item For n=4: $T_0 = 1.0$, $\alpha = 0.995$, max iterations = 50,000
    \item For n=8: $T_0 = 2.0$, $\alpha = 0.998$, max iterations = 200,000
\end{itemize}

\begin{table}[h]
\centering
\tiny
\begin{tabular}{lrrr}
\toprule
\textbf{Problem} & \textbf{Trial} & \textbf{Expansions} & \textbf{Time (s)} \\
\midrule
\multirow{3}{*}{n=4} & 1 & 68 & 0.0001 \\
& 2 & 101 & 0.0002 \\
& 3 & 42 & 0.0001 \\
\cmidrule{2-4}
& \textbf{Avg} & \textbf{70.33} & \textbf{0.0001} \\
\midrule
\multirow{3}{*}{n=8} & 1 & 791 & 0.0030 \\
& 2 & 1,451 & 0.0055 \\
& 3 & 571 & 0.0023 \\
\cmidrule{2-4}
& \textbf{Avg} & \textbf{937.67} & \textbf{0.0036} \\
\bottomrule
\end{tabular}
\caption{Simulated Annealing Results (100\% success)}
\end{table}

\textbf{Analysis:} SA solved all the test cases. The algorithm's ability to accept worse moves early on (when $T$ is high) lets it escape local optima. As the temperature goes down, it becomes more and more greedy.

\subsection{Genetic Algorithm Parameter Tuning}

GA evolves a population of solutions through selection, crossover, and mutation. I tested three different configurations on n=4 to find the best parameters.

\begin{table}[h]
\centering
\tiny
\begin{tabular}{lcccccc}
\toprule
\textbf{Cfg} & \textbf{Pop} & \textbf{CX} & \textbf{Mut} & \textbf{Gens} & \textbf{Avg Gen} & \textbf{Time (s)} \\
\midrule
A & 40 & 0.9 & 0.10 & 200 & 4.00 & 0.0006 \\
B & 80 & 0.9 & 0.05 & 300 & 0.67 & 0.0003 \\
C & 60 & 0.8 & 0.15 & 300 & 2.67 & 0.0007 \\
\bottomrule
\end{tabular}
\caption{GA Configuration Comparison on n=4}
\end{table}

\textbf{Configuration B Selected:} The largest population (80) with high crossover (0.9) and low mutation (0.05) worked best - it converged the fastest and had the lowest time.

\subsection{GA Results with Best Configuration}

\begin{table}[h]
\centering
\tiny
\begin{tabular}{lrccrc}
\toprule
\textbf{Prob} & \textbf{Trial} & \textbf{Success} & \textbf{Gens} & \textbf{Evals} & \textbf{Time (s)} \\
\midrule
\multirow{3}{*}{n=4} & 1 & Yes & 1 & 160 & 0.0004 \\
& 2 & Yes & 1 & 160 & 0.0005 \\
& 3 & Yes & 0 & 80 & 0.0001 \\
\cmidrule{2-6}
& \textbf{Avg} & \textbf{100\%} & \textbf{0.67} & \textbf{133} & \textbf{0.0003} \\
\midrule
\multirow{3}{*}{n=8} & 1 & Yes & 15 & 1,280 & 0.0084 \\
& 2 & Yes & 163 & 13,120 & 0.0890 \\
& 3 & Yes & 42 & 3,440 & 0.0226 \\
\cmidrule{2-6}
& \textbf{Avg} & \textbf{100\%} & \textbf{73.33} & \textbf{5,947} & \textbf{0.0400} \\
\bottomrule
\end{tabular}
\caption{GA Results (pop=80, cx=0.9, mut=0.05)}
\end{table}

\textbf{Analysis:} GA had a 100\% success rate on both problem sizes. The variation in generations for n=8 (15 to 163) shows how GA is stochastic in nature - you get different results each time.

\subsection{SA vs GA Comparison}

\begin{table}[h]
\centering
\tiny
\begin{tabular}{lccccl}
\toprule
\textbf{Alg} & \textbf{n=4 Evals} & \textbf{n=4 Time} & \textbf{n=8 Evals} & \textbf{n=8 Time} & \textbf{Success} \\
\midrule
SA & 70 & 0.0001s & 938 & 0.0036s & 100\% \\
GA & 133 & 0.0003s & 5,947 & 0.0400s & 100\% \\
\bottomrule
\end{tabular}
\caption{SA vs GA Performance Comparison}
\end{table}

\textbf{Discussion:} SA was more efficient - it needed fewer evaluations and less time. But GA's population-based approach has some advantages: it could be parallelized better, it's more robust to parameter choices, and it can keep track of multiple good solutions at once.

\section{Part D: Heuristic Search (4.0 pts)}

\subsection{A* Algorithm}

A* is a best-first search that uses the evaluation function $f(n) = g(n) + h(n)$ where $g(n)$ is the actual cost from the start to $n$ and $h(n)$ is the estimated cost from $n$ to the goal. A* is optimal if $h(n)$ is admissible (never overestimates).

\subsection{Heuristic Functions}

\textbf{h1: Misplaced Tiles} Counts how many tiles aren't in their goal position (not counting the blank).

\textbf{Admissible?} Yes. Each misplaced tile needs at least one move to get to its goal position, so $h_1(n) \leq h^*(n)$ where $h^*(n)$ is the true optimal cost.

\textbf{Consistent?} Yes. Moving a tile can only change the misplaced count by at most ±1, which is the same as the action cost. Formally: $h(n) \leq c(n,a,n') + h(n')$ where $c(n,a,n') = 1$.

\textbf{h2: Manhattan Distance} Sum of Manhattan distances (taxicab distance) for all tiles: $\sum |x_{current} - x_{goal}| + |y_{current} - y_{goal}|$.

\textbf{Admissible?} Yes. Manhattan distance is the minimum number of moves needed if no other tiles were blocking the path, so $h_2(n) \leq h^*(n)$.

\textbf{Consistent?} Yes. Moving a tile changes its Manhattan distance by exactly ±1 (the action cost): $h(n) \leq 1 + h(n')$.

\textbf{Dominance} $h_2$ dominates $h_1$: For any state $n$, $h_2(n) \geq h_1(n)$. This is because if a tile is misplaced ($h_1$ adds 1), its Manhattan distance is at least 1. A dominating heuristic expands fewer nodes because it gives better estimates.

\subsection{A* Experimental Results}

I tested both heuristics on the same three puzzle instances so the comparison would be fair.

\begin{table}[h]
\centering
\tiny
\begin{tabular}{crrr|rrr}
\toprule
& \multicolumn{3}{c}{\textbf{h1: Misplaced}} & \multicolumn{3}{c}{\textbf{h2: Manhattan}} \\
\cmidrule(lr){2-4} \cmidrule(lr){5-7}
\textbf{Trial} & \textbf{Moves} & \textbf{Exp} & \textbf{Time} & \textbf{Moves} & \textbf{Exp} & \textbf{Time} \\
\midrule
1 & 20 & 3,116 & 0.0132 & 20 & 433 & 0.0020 \\
2 & 18 & 1,254 & 0.0047 & 18 & 197 & 0.0008 \\
3 & 12 & 99 & 0.0004 & 12 & 18 & 0.0002 \\
\midrule
\textbf{Avg} & \textbf{16.67} & \textbf{1,490} & \textbf{0.0061} & \textbf{16.67} & \textbf{216} & \textbf{0.0010} \\
\bottomrule
\end{tabular}
\caption{A* Performance with Both Heuristics (8-puzzle)}
\end{table}

\subsection{Analysis}

\textbf{Optimality:} Both heuristics found optimal solutions (16.67 moves on average) which is what you'd expect from A* with admissible heuristics.

\textbf{Efficiency:} Manhattan distance expanded 6.9× fewer nodes (216 vs 1,490) and was 6.1× faster, which clearly shows how much better a stronger heuristic is.

\textbf{Comparison to Uninformed Search:} BFS (uninformed) expanded 1,234 nodes on average. A* with Manhattan only expanded 216 nodes (5.7× more efficient). Both guarantee optimal solutions, but A* is way more efficient.

\textbf{Why Manhattan is Better:} The misplaced tiles heuristic is too simple - it treats a tile one move away the same as a tile five moves away. Manhattan distance actually captures the minimum distance, so it gives much better guidance toward the goal.

\section{Conclusion}

This assignment helped me understand the fundamental AI search algorithms:

\textbf{Uninformed Search:} BFS guarantees the optimal solution but uses a lot of memory. DFS is memory-efficient but gives really bad solutions. IDS combines the good parts of both. BDS is the most efficient when you know what the goal is.

\textbf{Local Search:} SA and GA both successfully solve n-queens but in different ways. SA is simpler and faster, while GA offers better parallelization and keeps more diversity in the solutions.

\textbf{Informed Search:} A* with a good heuristic (like Manhattan distance) performs way better than uninformed search while still guaranteeing optimal solutions. The quality of your heuristic directly affects how efficient the algorithm is.

The experimental results matched what the theory predicted, which confirms that choosing the right algorithm depends on the problem characteristics, what resources you have available, and what kind of solution you need.

\section*{AI Use Disclosure}

I used AI tools (Claude.ai and ChatGPT 5.0 Thinking) to help me complete this assignment. Here's how I used them:

\begin{itemize}
    \item \textbf{Understanding the algorithms:} I used AI to help me understand how each search algorithm actually works, since this was my first time ever attempting to implement them.
    \item \textbf{Developing the code:} I learned about the algorithms from class slides and the textbook (\textit{Artificial Intelligence: A Modern Approach} by Russell and Norvig), I commented on where I found them in my code applicable, then wrote my own implementations. When I got stuck or had bugs, I asked AI for help troubleshooting.
    \item \textbf{Adding features:} AI helped me add things like time limits, random puzzle generation, running multiple trials automatically, and formatting the output nicely.
    \item \textbf{The PowerShell script:} ChatGPT wrote the \texttt{run\_all.ps1} script that runs all my programs and saves the output to a log file.
    \item \textbf{Writing the report:} Claude helped me analyze my experimental results and format everything in AAAI24 LaTeX format since I am still not an expert on using LaTeX and that is what AI is good at.
\end{itemize}

I want to be clear that I reviewed everything the AI gave me before I used it. I made sure I understood what the code was doing and that it actually worked correctly. Since I am not an expert on the algorithms, I made sure to write (and edited by Claude) good comments so that in the future I can look back and be reminded of what I did.  The AI didn't just do the whole assignment for me - I had to understand it, test it, and make decisions about what to include.  The 15 puzzles are a good example, I had to implement the code that set guardrails on how long the code would run so it wouldn't just burn out my computer and waste energy.

\section{Code Output Examples}

Here are some screenshots showing how the algorithms performed:

\subsection{n-Puzzle Algorithm Results}

\begin{figure}[h]
\centering
\includegraphics[width=0.85\columnwidth]{astar_3x3.jpg}
\caption{A* Heuristic Comparison: Manhattan expands 433 nodes vs Misplaced expands 3,116 nodes}
\end{figure}

\begin{figure}[h]
\centering
\includegraphics[width=0.85\columnwidth]{bds_3x3.jpg}
\caption{BDS on 8-puzzle: 14 moves in only 161 expansions}
\end{figure}

\begin{figure}[h]
\centering
\includegraphics[width=0.85\columnwidth]{bds_4x4.jpg}
\caption{BDS on 15-puzzle: 6 moves in only 19 expansions (really efficient)}
\end{figure}

\begin{figure}[h]
\centering
\includegraphics[width=0.85\columnwidth]{bfs_3x3.jpg}
\caption{BFS on 8-puzzle: 10 moves (optimal solution)}
\end{figure}

\begin{figure}[h]
\centering
\includegraphics[width=0.85\columnwidth]{bfs_4x4.jpg}
\caption{BFS on 15-puzzle: 6 moves with 114 nodes expanded}
\end{figure}

\begin{figure}[h]
\centering
\includegraphics[width=0.85\columnwidth]{dfs_3x3.jpg}
\caption{DFS on 8-puzzle: 56,106 moves (really bad)}
\end{figure}

\begin{figure}[h]
\centering
\includegraphics[width=0.85\columnwidth]{dfs_4x4.jpg}
\caption{DFS on 15-puzzle: Hit 10M node cap after 33 seconds}
\end{figure}

\begin{figure}[h]
\centering
\includegraphics[width=0.85\columnwidth]{ids_3x3.jpg}
\caption{IDS on 8-puzzle: Optimal 26-move solution in 12,892 expansions}
\end{figure}

\begin{figure}[h]
\centering
\includegraphics[width=0.85\columnwidth]{ids_4x4.jpg}
\caption{IDS on 15-puzzle: 4 moves with only 38 expansions}
\end{figure}

\subsection{n-Queens Algorithm Results}

\begin{figure}[h]
\centering
\includegraphics[width=0.85\columnwidth]{queens_ga_config_b.jpg}
\caption{Genetic Algorithm Configuration B testing on n=4 (best configuration)}
\end{figure}

\begin{figure}[h]
\centering
\includegraphics[width=0.85\columnwidth]{queens_ga_config_c.jpg}
\caption{Genetic Algorithm Configuration C testing on n=4}
\end{figure}

\begin{figure}[h]
\centering
\includegraphics[width=0.85\columnwidth]{queens_ga_n4.jpg}
\caption{Genetic Algorithm solving 4-queens with best configuration}
\end{figure}

\begin{figure}[h]
\centering
\includegraphics[width=0.85\columnwidth]{queens_ga_n8.jpg}
\caption{Genetic Algorithm solving 8-queens (163 generations for this trial)}
\end{figure}

\begin{figure}[h]
\centering
\includegraphics[width=0.85\columnwidth]{queens_sa_n4.jpg}
\caption{Simulated Annealing solving 4-queens in 101 expansions}
\end{figure}

\begin{figure}[h]
\centering
\includegraphics[width=0.85\columnwidth]{queens_sa_n8.jpg}
\caption{Simulated Annealing solving 8-queens in 1,451 expansions}
\end{figure}

\end{document}